\hypertarget{ch:conclusion}{%
\chapter{Conclusion}\label{ch:conclusion}}

\hypertarget{theoretical-contributions}{%
\section{Theoretical Contributions}\label{sec:theoretical-contributions}}

As previewed in Section~\ref{sec:contribution-preview}, this thesis makes three contributions. Having now traversed the full arc--phenomenon identification (Chapter~1), mechanism theorization (Chapter~2), empirical operationalization (Chapter~3), hypothesis testing (Chapter~4), and design principles (Chapter~5)--I review each contribution with the precision that evidence affords:

\textbf{First}, I introduce the \emph{golden cage} mechanism, explaining how funding constrains growth through governance homogeneity rather than moral hazard. The mechanism integrates \citeauthor{vandensteen2010interpersonal}'s sorting equilibrium, \citeauthor{eisenberg1984ambiguity}'s strategic ambiguity, and \citeauthor{ghemawat1991commitment}'s commitment analysis.

\textbf{Second}, I distinguish vision-level commitment from operational commitment. This explains why well-funded ventures diverge: some (Tesla) preserve flexibility through vision commitment; others (Better Place) foreclose it through operational commitment.

\textbf{Third}, I formalize caged learning (Theorem 1), showing how funding endogenously produces conditions (high \ensuremath{\mu}, low B) that prevent founders from updating their beliefs.

\hypertarget{practical-implications}{%
\section{Practical Implications}\label{practical-implications}}

The golden cage mechanism has distinct implications for founders, investors, and scholars.

For founders, the prescription is to design for adaptation before funding eliminates the skeptics who make adaptation possible. This means articulating vision at the direction level (``sustainable transport'') rather than the operational level (``battery swapping''). It means recruiting board members who share your view on \emph{why} but differ on \emph{how}. And it means sequencing capital sources so that flexibility survives long enough for market signals to clarify.

For investors, the prescription is to fund platform capability rather than product specificity. The ventures that succeed are those that reposition when market signals change. A founder who has never changed direction in three years may be perfectly calibrated--or structurally trapped. The question to ask: ``Who in your governance would advocate for pivoting if signals turn negative?''

For scholars, the prescription is to measure governance directly. This thesis documents behavioral patterns consistent with governance homogeneity, but the mechanism remains inferred rather than observed. Future research must measure board belief diversity through surveys, voting records, or text analysis of investor communications.

For VCs at the portfolio level, the industry-level correlations reveal a structural challenge: capital-intensive sectors like Hardware ($\rho = -0.11$) and Transportation ($\rho = -0.10$) systematically underperform relative to their funding levels. This pattern helps explain why VCs historically underweight these sectors despite their importance for climate, infrastructure, and industrial transformation. \citet{nanda2016financing} provide a complementary explanation grounded in real options: software ventures offer lower experimentation costs \emph{and} larger step-ups in value conditional on success---a more favorable payoff-to-cost ratio that makes early-stage software investments more valuable as real options. This asymmetry shifted VC portfolio composition toward software, with hardware, biotech, and energy declining substantially as shares of VC investment over the past two decades \citep{nanda2016financing}. Even as experimentation costs have fallen in other sectors (gene sequencing, 3D printing), software's combination of cheap experiments and high step-ups persists. The cage mechanism adds a governance dimension to this logic: even holding experimentation costs constant, capital-intensive sectors bind founders to specific architectures that preclude pivoting when early experiments reveal the need to change direction. The underperformance is not inevitable; it reflects the cage mechanism binding tighter where switching costs are high. VCs seeking exposure to capital-intensive sectors should pair funding with governance interventions: board diversity requirements, staged commitment structures, and explicit pivot optionality. Without such interventions, overweighting Hardware and Mobility may systematically underperform Software portfolios where the cage barely binds ($\rho = -0.001$) and cheap experiments enable Bayesian updating \citep{nanda2016financing}.

For policy makers, the implication is that capital-intensive sectors need more than capital---they need \emph{flexibility infrastructure}. Traditional industrial policy provides grants, subsidies, and loans to overcome funding gaps. But if the cage mechanism is correct, more capital alone may worsen outcomes by tightening governance homogeneity. Effective policy for Hardware, Transportation, and CleanTech should include: (1) \emph{pivot grants} that reward strategic adaptation rather than milestone adherence; (2) \emph{staged commitment structures} that preserve optionality longer; (3) \emph{diverse board requirements} as conditions for public funding; and (4) \emph{bridge programs} that allow ventures to reposition without losing momentum. The goal is not to replace VC funding but to complement it with mechanisms that counteract the cage. Quantum computing's positive correlation ($\rho = +0.095$) suggests that in pre-paradigmatic sectors, capital enables rather than constrains---policy should distinguish sectors where the cage binds from those where it releases.

\hypertarget{limitations}{%
\section{Limitations}\label{limitations}}

Four limitations warrant acknowledgment:

\textbf{First}, I document correlation, not causation. An alternative explanation remains: rigid founders attract more funding. I address this through three layers of defense: (1) selection is part of the mechanism; (2) conditioning on observables reduces selection; (3) future quasi-experimental approaches could provide identification.

\textbf{Second}, PitchBook overrepresents technology ventures in the United States. Generalization requires replication in other sectors and geographies.

\textbf{Third, and most critically, I infer governance homogeneity from behavioral outcomes (low repositioning), not direct measurement.} I derive the core claim, "governance lacks skeptics", from observing that well-funded ventures reposition less frequently. This behavioral pattern is \emph{consistent with} the sorting mechanism, but the mechanism itself remains unobserved.

Theory grounds this inference. \citeauthor{vandensteen2010interpersonal}'s \citeyearpar{vandensteen2010interpersonal} sorting equilibrium predicts that founders and investors with heterogeneous priors will sort into organizations led by like-minded others, a mathematical result, not an empirical claim. Applied to venture governance, this predicts belief convergence among board members. However, the prediction remains \emph{indirect}: I observe the predicted \emph{consequence} (low repositioning) rather than the posited \emph{cause} (belief homogeneity). The gap matters.

Future work must directly measure board composition diversity. Three approaches merit consideration: (a) survey-based measurement of founder-investor disagreement on strategic direction, (b) analysis of board voting records on strategic pivots, and (c) text analysis of investor communications (e.g., board meeting minutes, investor letters) to quantify belief divergence. Without such direct measurement, the governance homogeneity mechanism, while theoretically compelling and empirically consistent, remains a well-supported conjecture rather than established fact.

\textbf{Fourth}, the Funding-Growth Paradox (H3: $\rho(E,G) < 0$) is sensitive to how M\&A outcomes are coded. The baseline analysis codes M\&A exits as non-growth ($G = 0$) because the thesis studies organic scaling via Later Stage VC, not exit events. However, when M\&A is coded as success ($G = 1$), the correlation reverses from $-0.041$ to $+0.146$. This sensitivity does not affect the Commitment Cage (H1) or Flexibility Flex (H2), which are robust across all coding approaches (see Appendix~\ref{app:c}, Table~\ref{tab:robustness-summary}). The baseline coding is defensible---most M\&A represent acqui-hires or asset sales rather than billion-dollar exits---but the sensitivity warrants acknowledgment.

\hypertarget{alternative-explanations}{%
\subsection{Alternative Explanations}\label{alternative-explanations}}

The governance homogeneity mechanism proposed in this thesis competes with several alternative explanations for the funding-growth paradox. I consider three prominent alternatives and discuss why the evidence favors the governance account.

\begin{longtable}[]{@{}
  >{\raggedright\arraybackslash}p{(\columnwidth - 6\tabcolsep) * \real{0.2444}}
  >{\raggedright\arraybackslash}p{(\columnwidth - 6\tabcolsep) * \real{0.2667}}
  >{\raggedright\arraybackslash}p{(\columnwidth - 6\tabcolsep) * \real{0.2222}}
  >{\raggedright\arraybackslash}p{(\columnwidth - 6\tabcolsep) * \real{0.2667}}@{}}
\caption{Alternative Explanations vs.~Governance Homogeneity }\label{tab:alternatives}\tabularnewline
\toprule
\begin{minipage}[b]{\linewidth}\raggedright
Mechanism
\end{minipage} & \begin{minipage}[b]{\linewidth}\raggedright
Prediction
\end{minipage} & \begin{minipage}[b]{\linewidth}\raggedright
Evidence
\end{minipage} & \begin{minipage}[b]{\linewidth}\raggedright
Assessment
\end{minipage} \\
\midrule
\endfirsthead
\toprule
\begin{minipage}[b]{\linewidth}\raggedright
Mechanism
\end{minipage} & \begin{minipage}[b]{\linewidth}\raggedright
Prediction
\end{minipage} & \begin{minipage}[b]{\linewidth}\raggedright
Evidence
\end{minipage} & \begin{minipage}[b]{\linewidth}\raggedright
Assessment
\end{minipage} \\
\midrule
\endhead
\textbf{Moral Hazard} & Well-funded founders exert less effort & Should affect all ventures equally & Inconsistent: founders report \emph{wanting} to pivot but lacking board support \\
\textbf{Milestone Pressure} & Milestone-tied funding forces rigid execution & Movers should underperform (they miss milestones) & Inconsistent: Movers outperform 2.60\ensuremath{\times} despite milestone deviation \\
\textbf{Burn-rate Discipline} & High burn reduces runway for experimentation & Capital-light sectors should show weaker cage & Partially consistent: Software shows \ensuremath{\rho} \ensuremath{\approx} 0, but pattern reverses in Quantum \\
\textbf{Governance Homogeneity} & Like-minded investors filter skeptics & Cage binds where switching costs are high & Consistent: Hardware/Transportation show strongest cage effects \\
\bottomrule
\end{longtable}


Three alternatives warrant examination. \textbf{Moral Hazard} predicts reduced founder effort after funding, yet founders of failed well-funded ventures frequently express regret at not pivoting earlier---if moral hazard drove the pattern, founders would report satisfaction with their strategic persistence, not regret about rigidity. \textbf{Milestone Pressure} predicts that ventures deviating from milestones (Movers) should face capital constraints and underperform, but the opposite obtains: Movers achieve 2.60$\times$ higher success rates than Stayers, suggesting the benefit of strategic adaptation outweighs the cost of milestone deviation. \textbf{Burn-rate Discipline} predicts the cage should weaken in capital-light sectors, and Software's near-zero correlation ($\rho = -0.001$) is consistent with this account---however, Quantum's positive correlation ($\rho = +0.095$) is inconsistent, since if burn-rate were decisive, capital-intensive Quantum should show the strongest cage, not the weakest. The uncertainty-contingent pattern favors governance homogeneity over these alternatives as the primary mechanism.

\hypertarget{future-research}{%
\section{Future Research}\label{future-research}}

Three directions merit investigation. First, directly measure belief homogeneity through surveys or text analysis of board communications. Second, pursue quasi-experimental identification using VC fund vintage effects, geographic shocks, or industry funding cycles as instruments. Third, test governance interventions through field experiments on skeptic preservation strategies.

\hypertarget{closing}{%
\section{Closing}\label{closing}}

This thesis began with a paradox: the venture capital industry deploys hundreds of billions of dollars annually to fuel startup growth \citep{nvca2024}, yet early-stage funding correlates negatively with later-stage success. The paradox resolves through a simple mechanism--the golden cage. To secure capital, founders must commit to specific strategies. These commitments attract investors who believe in those strategies; skeptics never join. The resulting board contains only believers. When market signals suggest pivoting, no one advocates for change. The venture is trapped--not for lack of capital, but for lack of diverse perspectives. The evidence confirms this: funding suppresses repositioning ($\rho = -0.133$), repositioning predicts success (Movers outperform Stayers by 2.60$\times$), and the cage binds tightest in capital-intensive sectors where switching costs are high.

The cage need not be fatal. Founders can dance with it by committing at the vision level (preserving diverse believers), diagnosing bottlenecks before scaling (avoiding off-diagonal traps), and sequencing capital sources (building credibility before homogenizing governance). These design principles do not eliminate the paradox; they navigate it. The central insight is simple: resources and flexibility are not substitutes. Raising money does not buy the ability to change direction--if anything, it reduces that ability by filtering out the skeptics who would advocate for change. The ventures that survive are not those with the most resources, but those that preserve the governance diversity to act on what they learn.

For founders facing uncertainty, the prescription is counterintuitive: commit to the capacity to reposition, not to any particular position. Build governance that can challenge your assumptions. Design for adaptation before funding eliminates the skeptics who make adaptation possible. \textbf{Dance with the cage. Move to grow.}