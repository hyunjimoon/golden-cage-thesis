Startups fail not for lack of resources, but for lack of strategic flexibility. A puzzle emerges: ventures that raise more \fundinggreen{early funding} succeed less often later.

Funding homogenizes governance. Founders commit to specific strategies; investors who believe fund, skeptics self-select out. The resulting board lacks cognitive diversity to advocate pivots. I formalize this as the \cage{golden cage}: founders cannot pivot because governance lacks advocates for alternatives.

Analyzing 168,011 U.S. ventures, I document two regularities. First, funding suppresses repositioning. Second, \flex{repositioning predicts success}: movers outperform stayers by 2.6 times. The cage binds tightest in capital-intensive sectors but releases where no dominant design constrains exploration.

Three contributions emerge. Empirically, the funding-growth paradox holds as a robust regularity. Theoretically, belief-based sorting explains why. Prescriptively, founders can escape through the 3S framework: \textbf{Scope} (commit to vision, not solution), \textbf{Synchronized} (diagnose bottlenecks before scaling), and \textbf{Sequencing} (delay governance lock-in through staged capital).
