% ============================================
% COLOR DEFINITIONS FOR GOLDEN CAGE THESIS
% ============================================
% Usage: % ============================================
% COLOR DEFINITIONS FOR GOLDEN CAGE THESIS
% ============================================
% Usage: % ============================================
% COLOR DEFINITIONS FOR GOLDEN CAGE THESIS
% ============================================
% Usage: % ============================================
% COLOR DEFINITIONS FOR GOLDEN CAGE THESIS
% ============================================
% Usage: \input{latex(thesis)_color_definitions.tex} in preamble
%
% To disable colors (e.g., for thesis submission):
%   Change \colortermstrue to \colortermsfalse
% ============================================

\usepackage{xcolor}

% Color toggle switch
\newif\ifcolorterms
\colortermstrue  % Set to \colortermsfalse for black-only output

% Define colors based on toggle
\ifcolorterms
  % Paradox = Green (H1: overall phenomenon)
  \definecolor{ParadoxColor}{RGB}{40,167,69}
  % Cage = Red (H2: constraint, suppression)
  \definecolor{CageColor}{RGB}{220,53,69}
  % Flex = Blue (H3: flexibility, growth)
  \definecolor{FlexColor}{RGB}{0,123,255}
\else
  % All black for formal submission
  \definecolor{ParadoxColor}{RGB}{0,0,0}
  \definecolor{CageColor}{RGB}{0,0,0}
  \definecolor{FlexColor}{RGB}{0,0,0}
\fi

% ============================================
% SEMANTIC MACROS
% ============================================
% Use these instead of raw \textcolor for consistency

% H1: Funding-Growth Paradox (green)
\newcommand{\paradox}[1]{\textcolor{ParadoxColor}{\textbf{#1}}}

% H2: Commitment Cage (red)
\newcommand{\cage}[1]{\textcolor{CageColor}{\textbf{#1}}}

% H3: Flexibility Flex (blue)
\newcommand{\flex}[1]{\textcolor{FlexColor}{\textbf{#1}}}

% ============================================
% COLOR LEGEND (for reference)
% ============================================
% Green (ParadoxColor): Funding-Growth Paradox, H1, rho(E,G)
% Red (CageColor): Commitment Cage, H2, rho(E,R), suppression
% Blue (FlexColor): Flexibility Flex, H3, Mover Advantage, growth
 in preamble
%
% To disable colors (e.g., for thesis submission):
%   Change \colortermstrue to \colortermsfalse
% ============================================

\usepackage{xcolor}

% Color toggle switch
\newif\ifcolorterms
\colortermstrue  % Set to \colortermsfalse for black-only output

% Define colors based on toggle
\ifcolorterms
  % Paradox = Green (H1: overall phenomenon)
  \definecolor{ParadoxColor}{RGB}{40,167,69}
  % Cage = Red (H2: constraint, suppression)
  \definecolor{CageColor}{RGB}{220,53,69}
  % Flex = Blue (H3: flexibility, growth)
  \definecolor{FlexColor}{RGB}{0,123,255}
\else
  % All black for formal submission
  \definecolor{ParadoxColor}{RGB}{0,0,0}
  \definecolor{CageColor}{RGB}{0,0,0}
  \definecolor{FlexColor}{RGB}{0,0,0}
\fi

% ============================================
% SEMANTIC MACROS
% ============================================
% Use these instead of raw \textcolor for consistency

% H1: Funding-Growth Paradox (green)
\newcommand{\paradox}[1]{\textcolor{ParadoxColor}{\textbf{#1}}}

% H2: Commitment Cage (red)
\newcommand{\cage}[1]{\textcolor{CageColor}{\textbf{#1}}}

% H3: Flexibility Flex (blue)
\newcommand{\flex}[1]{\textcolor{FlexColor}{\textbf{#1}}}

% ============================================
% COLOR LEGEND (for reference)
% ============================================
% Green (ParadoxColor): Funding-Growth Paradox, H1, rho(E,G)
% Red (CageColor): Commitment Cage, H2, rho(E,R), suppression
% Blue (FlexColor): Flexibility Flex, H3, Mover Advantage, growth
 in preamble
%
% To disable colors (e.g., for thesis submission):
%   Change \colortermstrue to \colortermsfalse
% ============================================

\usepackage{xcolor}

% Color toggle switch
\newif\ifcolorterms
\colortermstrue  % Set to \colortermsfalse for black-only output

% Define colors based on toggle
\ifcolorterms
  % Paradox = Green (H1: overall phenomenon)
  \definecolor{ParadoxColor}{RGB}{40,167,69}
  % Cage = Red (H2: constraint, suppression)
  \definecolor{CageColor}{RGB}{220,53,69}
  % Flex = Blue (H3: flexibility, growth)
  \definecolor{FlexColor}{RGB}{0,123,255}
\else
  % All black for formal submission
  \definecolor{ParadoxColor}{RGB}{0,0,0}
  \definecolor{CageColor}{RGB}{0,0,0}
  \definecolor{FlexColor}{RGB}{0,0,0}
\fi

% ============================================
% SEMANTIC MACROS
% ============================================
% Use these instead of raw \textcolor for consistency

% H1: Funding-Growth Paradox (green)
\newcommand{\paradox}[1]{\textcolor{ParadoxColor}{\textbf{#1}}}

% H2: Commitment Cage (red)
\newcommand{\cage}[1]{\textcolor{CageColor}{\textbf{#1}}}

% H3: Flexibility Flex (blue)
\newcommand{\flex}[1]{\textcolor{FlexColor}{\textbf{#1}}}

% ============================================
% COLOR LEGEND (for reference)
% ============================================
% Green (ParadoxColor): Funding-Growth Paradox, H1, rho(E,G)
% Red (CageColor): Commitment Cage, H2, rho(E,R), suppression
% Blue (FlexColor): Flexibility Flex, H3, Mover Advantage, growth
 in preamble
%
% To disable colors (e.g., for thesis submission):
%   Change \colortermstrue to \colortermsfalse
% ============================================

\usepackage{xcolor}

% Color toggle switch
\newif\ifcolorterms
\colortermstrue  % Set to \colortermsfalse for black-only output

% Define colors based on toggle
\ifcolorterms
  % Paradox = Green (H1: overall phenomenon)
  \definecolor{ParadoxColor}{RGB}{40,167,69}
  % Cage = Red (H2: constraint, suppression)
  \definecolor{CageColor}{RGB}{220,53,69}
  % Flex = Blue (H3: flexibility, growth)
  \definecolor{FlexColor}{RGB}{0,123,255}
\else
  % All black for formal submission
  \definecolor{ParadoxColor}{RGB}{0,0,0}
  \definecolor{CageColor}{RGB}{0,0,0}
  \definecolor{FlexColor}{RGB}{0,0,0}
\fi

% ============================================
% SEMANTIC MACROS
% ============================================
% Use these instead of raw \textcolor for consistency

% H1: Funding-Growth Paradox (green)
\newcommand{\paradox}[1]{\textcolor{ParadoxColor}{\textbf{#1}}}

% H2: Commitment Cage (red)
\newcommand{\cage}[1]{\textcolor{CageColor}{\textbf{#1}}}

% H3: Flexibility Flex (blue)
\newcommand{\flex}[1]{\textcolor{FlexColor}{\textbf{#1}}}

% ============================================
% COLOR LEGEND (for reference)
% ============================================
% Green (ParadoxColor): Funding-Growth Paradox, H1, rho(E,G)
% Red (CageColor): Commitment Cage, H2, rho(E,R), suppression
% Blue (FlexColor): Flexibility Flex, H3, Mover Advantage, growth
