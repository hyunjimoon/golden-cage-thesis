\hypertarget{ch:data}{%
\chapter{Data and Identification}\label{ch:data}}

\hypertarget{introduction}{%
\section{Introduction}\label{sec:ch3-introduction}}

Chapter~\ref{ch:theory} argued that funding creates a golden cage by homogenizing governance beliefs. This chapter describes how I test that theory. The core challenge is measuring flexibility--a latent capability that cannot be directly observed. My solution is to measure its behavioral manifestation: repositioning, the observable shift in a venture's strategic positioning over time. Ventures that reposition reveal they had flexibility; ventures that hold position may lack it.

I analyze 168,011 U.S. ventures from PitchBook (2021--2025). To measure repositioning, I use dictionary-based text analysis of company descriptions, computing how much each venture's strategic breadth changed between funding rounds. The method draws on established research in category spanning \citep{zuckerman1999the} and linguistic concreteness \citep{pan2018corporate}.

A key identification concern is selection: high-conviction founders may both raise more capital and resist pivoting, producing correlation without causation. I address this in Section~\ref{identification-strategy}, arguing that selection is part of the mechanism (not a confound) and conditioning on observable characteristics.

\hypertarget{data-sources-and-sample-construction}{%
\section{Data Sources and Sample Construction}\label{data-sources-and-sample-construction}}

I construct a panel of 168,011 ventures from PitchBook, covering the period 2021--2025. PitchBook provides comprehensive coverage of U.S. venture-backed companies, including funding rounds, company descriptions, and outcome data.

\textbf{Sample Construction.} The initial universe contains 488,381 ventures. I filter to U.S.-headquartered ventures at early stage (Seed through Series B) with at least 24 months of observable history and complete data on core variables, yielding 168,011 ventures (34.4\% retention).

\hypertarget{variable-operationalization}{%
\section{Variable Operationalization}\label{variable-operationalization}}

 \begin{longtable}[]{@{}
    >{\centering\arraybackslash}p{(\columnwidth - 6\tabcolsep) * \real{0.1818}}
    >{\raggedright\arraybackslash}p{(\columnwidth - 6\tabcolsep) * \real{0.2273}}
    >{\raggedright\arraybackslash}p{(\columnwidth - 6\tabcolsep) * \real{0.1364}}
    >{\raggedright\arraybackslash}p{(\columnwidth - 6\tabcolsep) * \real{0.4545}}@{}}
  \caption{Variable Definitions and Causal Structure }\label{tab:variables}\tabularnewline
  \toprule
  \begin{minipage}[b]{\linewidth}\centering
  Symbol
  \end{minipage} & \begin{minipage}[b]{\linewidth}\raggedright
  Variable
  \end{minipage} & \begin{minipage}[b]{\linewidth}\raggedright
  Type
  \end{minipage} & \begin{minipage}[b]{\linewidth}\raggedright
  Operationalization
  \end{minipage} \\
  \midrule
  \endfirsthead
  \toprule
  \begin{minipage}[b]{\linewidth}\centering
  Symbol
  \end{minipage} & \begin{minipage}[b]{\linewidth}\raggedright
  Variable
  \end{minipage} & \begin{minipage}[b]{\linewidth}\raggedright
  Type
  \end{minipage} & \begin{minipage}[b]{\linewidth}\raggedright
  Operationalization
  \end{minipage} \\
  \midrule
  \endhead
  \textbf{C} & Commitment & Choice & Initial strategic specificity index (0--100): product category count, milestone granularity, funding structure \\
  \textbf{E} & Early Funding & Outcome & Early-stage capital secured (first\_financing\_size, M USD, log-transformed) \\
  \textbf{F} & Flexibility & Capacity & Governance-permitted change capacity (inferred from R) \\
  \textbf{B} & Strategic Breadth & State & Market positioning specificity (0--100 scale via dictionary-based vagueness) \\
  \textbf{R} & Repositioning & Action & $|B_T - B_0|$, magnitude of strategic change \\
  \textbf{G} & Growth & Outcome & Binary: $G = 1$ if reached Later Stage VC (Series C+); base rate 11.5\% \\
  \bottomrule
  \end{longtable}

\textbf{From latent flexibility to observable repositioning.} The theory chapter treats strategic flexibility ($F$) as a \emph{latent} capability: the ability to keep multiple viable paths live under uncertainty. Because $F$ is unobserved in administrative venture data, I proxy it with \textbf{repositioning} ($R$), the observable behavioral manifestation of latent flexibility. Ventures that reposition reveal that they retained (and were permitted by governance to exercise) flexibility; ventures that remain static may lack the capability or may be structurally caged. This proxy motivates the empirical focus on $E \rightarrow R$ and $R \rightarrow G$ in Chapter~\ref{ch:results}.

\hypertarget{strategic-breadth-b}{%
\subsection{Strategic Breadth (B)}\label{strategic-breadth-b}}

I operationalize \textbf{Strategic Breadth (B)} using dictionary-based text analysis of company descriptions. Drawing on category spanning research \citep{zuckerman1999the, pontikes2012two} and linguistic concreteness research \citep{pan2018corporate}, I construct a continuous measure (0--100) that captures the degree of vagueness in a venture's positioning.

The measure combines two components. The first is \emph{Categorical Vagueness}: how much a description uses broad, umbrella terms (``platform,'' ``ecosystem,'' ``solution'') rather than specific market categories (``mobile payments,'' ``enterprise SaaS''). A company describing itself as a ``technology platform'' spans many categories; one describing itself as ``inventory management software for small retailers'' does not. The second component is \emph{Concreteness Deficit}: whether the description lacks specific binding markers such as quantitative targets (``10,000 users'') or narrow technical specifications (``HIPAA-compliant cloud storage'').

The resulting score ranges from 0 to 100. A score of 0 means maximally specific positioning--the company has committed to a narrow path. A score of 100 means maximally vague positioning--the company retains many possible directions. The sample mean is B = 52.3 (SD = 18.4). Full construction details appear in Appendix A.

\textbf{Illustrative Examples.} Table~\ref{tab:breadth-examples} demonstrates how the breadth measure (B) captures strategic positioning using examples from the autonomous vehicle (AV) industry--a capital-intensive sector where the golden cage mechanism binds tightly. Movers repositioned substantially between 2021--2025; Stayers maintained consistent positioning.

\begin{table}[h]

\centering
\caption{Breadth Measure: Illustrative Examples from AV Industry}
\label{tab:breadth-examples}

\small

\begin{tabular}{p{2.2cm}p{4.2cm}p{4.2cm}cc}

\toprule

\textbf{Company} & \textbf{2021 Description} & \textbf{2025 Description} & $\Delta B$ & \textbf{Growth Mult.} \\

\midrule

\multicolumn{5}{l}{\emph{Panel A: Movers (Zoom-Out)}} \\

\addlinespace

\textbf{Aurora} & ``Developer of autonomous trucks for freight logistics'' & ``Autonomous driving platform powering trucking, ride-hailing, and delivery'' & +38.2 & \\

& (Specific: trucking focus) & (Broad: multi-modal platform) & & \\

& $B_0 = 42.1$ (precise) & $B_T = 80.3$ (vague) & & $3.2\times$ \\

\midrule

\multicolumn{5}{l}{\emph{Panel B: Movers (Zoom-In)}} \\

\addlinespace

\textbf{Cruise} & ``Developer of self-driving vehicles for urban mobility'' & ``Provider of personal autonomous vehicle technology for OEMs'' & $-35.6$ & \\

& (Broad: urban mobility) & (Specific: OEM licensing) & & \\

& $B_0 = 76.4$ (vague) & $B_T = 40.8$ (precise) & & $2.9\times$ \\

\midrule

\multicolumn{5}{l}{\emph{Panel C: Stayers}} \\

\addlinespace

\textbf{Argo AI} & ``Developer of Level 4 autonomous driving technology for robotaxis'' & ``Developer of Level 4 autonomous driving technology for robotaxis'' & 0.0 & \\

& (Unchanged positioning despite market signals) & & & \\

& $B_0 = 58.2$ & $B_T = 58.2$ & & $1.0\times$* \\

\bottomrule

\end{tabular}

\end{table}

\emph{Notes: B = vagueness score (0--100 scale); $\Delta B = B_T - B_0$; Growth Multiple $= F_t/E$ (total subsequent funding / early funding). *Argo AI shut down in October 2022; growth multiple reflects capital consumed before shutdown.}

\textbf{Why AV?} The autonomous vehicle industry provides a natural laboratory for the golden cage mechanism because it combines three conditions that amplify the cage: (1) high capital intensity--\$100M+ funding rounds create strong sunk costs, (2) regulatory uncertainty--policy landscapes shift unpredictably across jurisdictions, and (3) technology path uncertainty--viable architectures compete (lidar vs. camera-only, robotaxi vs. personal AV, trucking vs. passenger). These conditions create strong investor sorting: VCs who fund AV ventures hold firm beliefs about which approach will win.

\textbf{Aurora (Zoom-Out, $3.2\times$):} Aurora began in 2021 as a trucking-focused company: ``autonomous trucks for freight logistics.'' By 2025, Aurora had broadened to a multi-modal ``Aurora Driver'' platform powering trucking, ride-hailing, and delivery. This repositioning--from specific application to general platform--attracted new partners (PACCAR, Volvo, Uber Freight) and enabled \$820M+ in additional funding. The zoom-out preserved optionality: if trucking unit economics proved unfavorable, the platform could pivot to other applications without abandoning the core technical asset.

\textbf{Cruise (Zoom-In, $2.9\times$):} Cruise took the opposite path. In 2021, Cruise positioned broadly as an ``urban mobility'' company operating robotaxis in San Francisco. By 2025, GM's ``Cruise 2.0'' strategy narrowed focus: exiting the robotaxi business to license autonomous technology to OEMs for personal vehicles. This zoom-in responded to market signals that robotaxi unit economics were challenging. The repositioning freed Cruise from fleet operations costs while monetizing its core technical capability through licensing.

\textbf{Argo AI (Stayer, $1.0\times$):} Argo AI maintained identical positioning from 2021 through its October 2022 shutdown: ``Level 4 autonomous driving technology for robotaxis.'' Despite \$3.6B in funding from Ford and Volkswagen, Argo failed to attract new investors when both backers reduced commitment. The company's inability to reposition illustrates the cage mechanism: Ford and VW, as thesis-driven investors, had funded a specific technical approach (robotaxis). When that approach encountered headwinds, Argo's board--populated solely by believers in robotaxis--lacked advocates for alternatives like trucking or OEM licensing. The cage was structural, not motivational.

\textbf{The Key Pattern:} Both Movers achieved $\sim$3$\times$ growth multiples; the Stayer achieved $1.0\times$ (capital consumed without subsequent growth). The \emph{direction} of repositioning differed--Aurora zoomed out, Cruise zoomed in--but \emph{movement itself} distinguished survivors from the caged. This pattern motivates the binary Mover/Stayer classification used throughout the empirical analysis.

\hypertarget{repositioning-r}{%
\subsection{Repositioning (R)}\label{repositioning-r}}

\textbf{Repositioning (R).} Repositioning magnitude measures the absolute change in strategic breadth: $R_i = |B_T - B_0|$, where $B_0$ is breadth at baseline (2021) and $B_T$ at endpoint (2025). Importantly, most ventures do not reposition at all: 61.2\% show R = 0 (I call these ``Stayers''), while only 38.8\% show R \textgreater{} 0 (``Movers''). This pattern (most ventures holding position) is consistent with the golden cage theory: governance constraints make repositioning difficult.

\textbf{Growth (G) and Growth Multiple.}

I use two outcome measures. \textbf{Growth (G)} is a binary indicator: $G = 1$ if the venture reached Later Stage VC (Series C or beyond) by the end of the observation window, $G = 0$ otherwise. The base growth rate is 11.0\%. The \textbf{Mover Advantage} (2.60$\times$) compares growth rates: $P(G=1|\text{Mover}) / P(G=1|\text{Stayer})$ = 17.6\% / 6.7\%. Separately, the \textbf{Growth Multiple} $= F_t/E$ measures continuous funding scale for illustrative cases. \emph{Robustness checks using alternative threshold definitions are provided in Appendix C.}

\hypertarget{descriptive-statistics}{%
\section{Descriptive Statistics}\label{descriptive-statistics}}

\begin{longtable}[]{@{}lrrrrr@{}}
\caption{Descriptive Statistics (N = 180,994) }\label{tab:descriptive}\tabularnewline
\toprule
Variable & Mean & SD & Min & Median & Max \\
\midrule
\endfirsthead
\toprule
Variable & Mean & SD & Min & Median & Max \\
\midrule
\endhead
Early Funding (E, M USD) & 4.2 & 8.7 & 0.1 & 1.5 & 250 \\
Strategic Breadth (\ensuremath{B_0}) & 52.3 & 18.4 & 0 & 51 & 100 \\
Strategic Breadth (B\_T) & 54.1 & 19.2 & 0 & 53 & 100 \\
Repositioning (R, standardized) & 0.31 & 0.42 & 0 & 0.15 & 2.8 \\
Growth (G = (F\_t-E)/E) & 0.67 & 2.0 & -29 & 0.09 & 265 \\
\bottomrule
\end{longtable}


\begin{figure}[htbp]
\centering
\includegraphics[width=0.9\textwidth]{img/Ch3_Fig1_distributions_E_B0.png}
\caption{Distributions of Early Funding ($E$) and Baseline Strategic Breadth ($B_0$). Early funding is right-skewed with median \$1.5M; strategic breadth shows bimodal distribution concentrated at high values.}
\label{fig:distributions-E-B0}
\end{figure}

\begin{figure}[htbp]
\centering
\includegraphics[width=0.9\textwidth]{img/Ch3_Fig2_distributions_R_G.png}
\caption{Distributions of Repositioning ($R$) and Growth ($G$). 61.2\% of ventures show no repositioning (Stayers); 11.0\% reach Later Stage VC (Growth = 1).}
\label{fig:distributions-R-G}
\end{figure}

The sample divides into Stayers (102,742 ventures, 61.2\%) who show no strategic movement, and Movers (65,269 ventures, 38.8\%) who shifted their strategic breadth by a measurable amount. Overall, 11.0\% of ventures reach Later Stage VC by the end of the observation window, with average strategic breadth at baseline of B = 45.2 (SD = 12.6).

\hypertarget{identification-strategy}{%
\section{Identification Strategy}\label{identification-strategy}}

The central challenge is distinguishing selection from treatment effects. High-conviction founders may both raise more capital and resist pivoting--not because funding caused rigidity, but because conviction drove both outcomes. This concern is valid, and I do not claim causal identification.

However, I argue that selection is part of the mechanism, not a confound to be eliminated. The golden cage theory predicts that funding and rigidity correlate \emph{because} committed founders attract committed investors through sorting. The selection concern is precisely what the theory describes. The question is not whether selection exists--it does--but whether the pattern is consistent with the theoretical mechanism.

I address identification in three ways. First, I condition on survival: all ventures in the comparison survived to Year 3, ensuring equal opportunity to reposition. Among these ventures, Movers still achieve 2.60$\times$ higher success rates than Stayers. Second, I control for observable characteristics: founder experience, industry fixed effects, cohort timing, and initial positioning. Third, I interpret results as robust correlational patterns consistent with theory, not as causal effects. Future work could exploit quasi-experimental variation (VC fund vintage effects, geographic funding shocks) to strengthen causal claims.

\hypertarget{conclusion}{%
\section{Conclusion}\label{conclusion}}

This chapter described how I test the cage hypotheses using 168,011 U.S. ventures from PitchBook (2021--2025), with repositioning measured through dictionary-based text analysis. The key finding is that repositioning is rare---only 40\% of ventures change their strategic positioning, consistent with the golden cage theory that governance constraints make movement difficult. The base success rate is 11.0\%, but this masks substantial heterogeneity between Movers and Stayers. Chapter~\ref{ch:results} tests whether funding suppresses repositioning (H1), whether repositioning predicts success (H2), and where these patterns vary across industries.