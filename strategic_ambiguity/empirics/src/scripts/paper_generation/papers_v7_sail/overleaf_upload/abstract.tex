Venture capital operates on the premise that early capital accelerates growth. Yet analyzing 168,011 startups, I document a \paradox{Funding-Growth Paradox}: early funding correlates \emph{negatively} with later-stage success. At the individual firm level, the correlation is modest ($\rho = -0.04$), reflecting high within-firm variance. At the industry level, however, the pattern is stark: capital-intensive sectors systematically underperform ($\rho = -0.11$ to $-0.10$). Why would resources hurt?

The answer lies in governance. To secure funding, founders commit to specific strategies. These commitments attract investors who believe in those strategies; skeptics self-select out. The resulting board lacks cognitive diversity. When market signals suggest pivoting, no one advocates for change. The venture is trapped---not for lack of capital, but for lack of diverse perspectives. I call this the \emph{golden cage}.

The paradox decomposes into two forces. First, the \cage{Commitment Cage}: funding suppresses repositioning ($\rho = -0.133$, $p < 0.001$) because governance homogeneity blocks adaptation. Second, the \flex{Flexibility Flex}: repositioning predicts success, with Movers outperforming Stayers by 2.60$\times$ (17.6\% vs.\ 6.7\%). Together: $dG/dE = (dG/dR) \times (dR/dE) = (+) \times (-) = (-)$. Funding suppresses the very mechanism required for survival.

Industry heterogeneity reveals boundary conditions. The cage binds tightest in capital-intensive sectors (Hardware: $\rho = -0.11$; Transportation: $\rho = -0.10$) where infrastructure investments create switching costs. It releases in pre-paradigmatic sectors (Quantum: $\rho = +0.10$) where no dominant design constrains architectural choices.

To escape the cage, founders can commit at the vision level (``sustainable transport'') rather than the operational level (``battery swapping infrastructure''). Vision-level commitment attracts diverse believers who agree on direction but differ on mechanisms---preserving the governance diversity required for adaptation. This thesis contributes a governance-based theory of venture rigidity and offers design principles for commitment structures that preserve flexibility.
