% =========================================================
% APPENDICES
% =========================================================

\chapter{Variable Construction Details}
\label{app:a}

\section{PitchBook Data Fields}

\begin{table}[htbp]
    \centering
    \caption{Primary Data Fields from PitchBook}
    \label{tab:pb-fields}
    \begin{tabular}{@{}llll@{}}
        \toprule
        \textbf{Field Name} & \textbf{Type} & \textbf{Description} & \textbf{Usage} \\
        \midrule
        \texttt{org\_uuid} & String & Unique venture identifier & Primary key \\
        \texttt{company\_description} & Text & Business description & Strategic Breadth ($B$) \\
        \texttt{primary\_industry} & Categorical & Industry classification & Heterogeneity analysis \\
        \texttt{first\_financing\_size} & Numeric & Initial funding (USD) & Early Funding ($E$) \\
        \texttt{last\_financing\_deal\_type} & Categorical & Most recent stage & Growth ($G$) \\
        \texttt{total\_raised} & Numeric & Cumulative funding & Growth Scale ($G$) \\
        \bottomrule
    \end{tabular}
\end{table}

\section{Vagueness Dictionary}

127 terms classified as:
\begin{itemize}
    \item \textbf{Vague (high B):} platform, ecosystem, solutions, enable, transform, optimize, leverage, innovative, next-generation, comprehensive, integrated, scalable
    \item \textbf{Specific (low B):} device, application, tool, product, service, system, manufacturer, operator, provider, developer
\end{itemize}

\section{Variable Definitions}

\subsection*{Strategic Breadth ($B$)}
\begin{equation}
B = 50 \times \frac{\text{vague\_terms}}{\max(\text{vague})} + 50 \times \left(1 - \frac{\text{concrete\_markers}}{\max(\text{concrete})}\right)
\end{equation}

\subsection*{Repositioning ($R$)}
\begin{equation}
R = \lvert B_T - B_0 \rvert
\end{equation}

\subsection*{Outcomes}
\begin{itemize}
    \item \textbf{Growth ($G$)}: Binary = 1 if reached Later Stage VC (Series C+). Base rate: 11.5\%.
    \item \textbf{Growth Multiple}: Continuous funding scale = $F_t / E$ (total subsequent funding / early funding). Used for illustrative cases only.
\end{itemize}


\chapter{Proof of Theorem 1}
\label{app:b}

\textbf{Theorem 1 (Caged Learning).} Learning ceases when $\mu(1 - \mu) < \varepsilon / B$.

\section{Proof}

\textbf{Step 1.} Value of information $\propto$ belief variance: $\mu(1-\mu)$.

\textbf{Step 2.} Signal quality $\propto 1/B$: Low $B \to$ precise; High $B \to$ noisy.

\textbf{Step 3.} Learning ceases when: $\mu(1-\mu) < \varepsilon/B$

\textbf{Step 4.} Early funding triggers both:
\begin{enumerate}
    \item Van den Steen sorting $\to$ high $\mu$
    \item Operational commitment $\to$ low $B$
\end{enumerate}

Combined: $\underbrace{\mu(1-\mu)}_{\downarrow} < \underbrace{\varepsilon/B}_{\uparrow}$ \hfill $\square$


\chapter{Robustness Tests}
\label{app:c}

This appendix systematically stress-tests the thesis's three core claims by varying each analytical choice that could influence results. Section~\ref{sec:robustness-framework} presents the framework. Sections~\ref{sec:robust-measurement}--\ref{sec:robust-sample} test each category. Section~\ref{sec:robustness-summary} summarizes all results.

% ============================================================================
\section{Framework: What Could Invalidate the Claims?}
\label{sec:robustness-framework}
% ============================================================================

The thesis makes three empirical claims:
\begin{enumerate}
    \item \textbf{H1 (Commitment Cage)}: $\rho(E,R) < 0$ --- funding suppresses repositioning
    \item \textbf{H2 (Flexibility Flex)}: $\rho(R,G) > 0$ --- repositioning predicts growth (Mover Advantage $> 1$)
    \item \textbf{H3 (Funding-Growth Paradox)}: $\rho(E,G) < 0$ --- funding correlates negatively with growth
\end{enumerate}

Each claim depends on analytical choices in three categories:

\begin{table}[htbp]
    \centering
    \caption{Analytical Choices and Stress Test Strategy}
    \label{tab:stress-test-strategy}
    \begin{tabular}{@{}p{2.5cm}p{4cm}p{5.5cm}@{}}
        \toprule
        \textbf{Category} & \textbf{Choice Made} & \textbf{Stress Test} \\
        \midrule
        \multicolumn{3}{l}{\textit{Measurement}} \\
        \quad Mover threshold & $R > 0$ (any repositioning) & Vary: $R > 1, 5, 10$; quartile crossing \\
        \quad Breadth scale & Raw score (0--100) & Rank normalization \\
        \addlinespace
        \multicolumn{3}{l}{\textit{Outcome definition}} \\
        \quad Growth ($G$) & Binary (Later Stage VC) & Continuous alternative ($K/E$) \\
        \quad M\&A coding & $G = 0$ (non-growth) & $G = 1$ (success); Censored (excluded) \\
        \addlinespace
        \multicolumn{3}{l}{\textit{Sample construction}} \\
        \quad Time window & 2021--2025 (4 years) & Cohort effects by founding year \\
        \quad Survival requirement & Requires $B_T$ (2025 data) & Bounds analysis for excluded firms \\
        \bottomrule
    \end{tabular}
\end{table}

% ============================================================================
\section{Measurement Robustness}
\label{sec:robust-measurement}
% ============================================================================

\subsection{Mover Threshold Sensitivity}

The baseline analysis classifies firms as Movers if $R > 0$ (any repositioning). This threshold is arbitrary. Table~\ref{tab:threshold} tests robustness to stricter definitions.

\begin{table}[htbp]
    \centering
    \caption{Mover Advantage by Threshold Definition}
    \label{tab:threshold}
    \begin{tabular}{@{}lrrrr@{}}
        \toprule
        \textbf{Definition} & \textbf{Movers \%} & \textbf{Stayer Succ} & \textbf{Mover Succ} & \textbf{Advantage} \\
        \midrule
        R $>$ 0 (Baseline) & 38.8\% & 6.7\% & 17.6\% & \textbf{2.60$\times$} \\
        R $>$ 1 & 13.6\% & 10.4\% & 18.0\% & 1.72$\times$ \\
        R $>$ 5 & 9.0\% & 10.8\% & 17.7\% & 1.63$\times$ \\
        R $>$ 10 & 6.3\% & 11.0\% & 17.9\% & 1.62$\times$ \\
        Quartile Cross & 6.4\% & 11.0\% & 17.6\% & 1.60$\times$ \\
        \bottomrule
    \end{tabular}
\end{table}

\textbf{Result}: Mover Advantage ranges 1.60$\times$--2.60$\times$ across definitions. \textbf{H2 is robust.}

\begin{figure}[htbp]
    \centering
    \includegraphics[width=0.9\textwidth]{img/AppC_Fig2_threshold_robustness.png}
    \caption{Mover Advantage across threshold definitions. The advantage persists (always $> 1$) regardless of how ``Mover'' is defined.}
    \label{fig:threshold}
\end{figure}

\subsection{Breadth Score Distribution}

The Strategic Breadth score ($B$) is bimodal, which affects the Mover/Stayer classification:

\begin{itemize}
    \item Low $B$ (0--50): 17.9\%
    \item Mid $B$ (50--75): 0.0\%
    \item High $B$ (75--100): 82.1\%
\end{itemize}

\begin{figure}[htbp]
    \centering
    \includegraphics[width=0.9\textwidth]{img/AppC_Fig1_bimodal_distribution.png}
    \caption{Bimodal $B_0$ distribution. Most firms cluster at high breadth (vague descriptions).}
    \label{fig:bimodal}
\end{figure}

\subsection{Rank Normalization}

Under rank normalization (converting $B$ to percentile ranks):
\begin{itemize}
    \item 100\% become ``Movers'' (each rank unique)
    \item $R > 0$ definition becomes uninformative
    \item Threshold-based definitions ($R > 5$, Quartile Cross) remain valid and show consistent Mover Advantage
\end{itemize}

\textbf{Result}: Core findings hold under rank normalization with threshold-based definitions.

% ============================================================================
\section{Outcome Definition Robustness}
\label{sec:robust-outcome}
% ============================================================================

\subsection{M\&A Outcome Coding Sensitivity}
\label{sec:ma-sensitivity}

The baseline analysis codes M\&A exits as non-growth ($G = 0$) because the thesis studies organic growth via Later Stage VC, not exit outcomes. However, M\&A could reasonably be coded as success ($G = 1$) or excluded entirely (censored). Table~\ref{tab:ma-sensitivity} tests all three approaches.

\begin{table}[htbp]
    \centering
    \caption{Sensitivity to M\&A Outcome Coding}
    \label{tab:ma-sensitivity}
    \begin{tabular}{@{}lrrrrrl@{}}
        \toprule
        \textbf{M\&A Coding} & \textbf{N} & \textbf{G rate} & $\rho(E,R)$ & $\rho(R,G)$ & $\rho(E,G)$ & \textbf{MA} \\
        \midrule
        Failure ($G=0$) & 172,260 & 10.6\% & $-0.130$ & $+0.180$ & $-0.041$ & 2.59$\times$ \\
        Success ($G=1$) & 172,260 & 29.6\% & $-0.130$ & $+0.016$ & $+0.146$ & 1.04$\times$ \\
        Censored & 139,641 & 13.1\% & $-0.098$ & $+0.177$ & $-0.002$ & 2.31$\times$ \\
        \bottomrule
    \end{tabular}
\end{table}

\begin{figure}[htbp]
    \centering
    \includegraphics[width=0.95\textwidth]{img/AppC_Fig3_ma_sensitivity.png}
    \caption{Robustness of core correlations across M\&A coding approaches.}
    \label{fig:ma-sensitivity}
\end{figure}

\textbf{Results}:
\begin{itemize}
    \item \textbf{H1 is robust}: $\rho(E,R) < 0$ under all codings
    \item \textbf{H2 is robust}: $\rho(R,G) > 0$ and Mover Advantage $> 1$ under all codings
    \item \textbf{H3 is NOT robust}: $\rho(E,G)$ flips from $-0.041$ to $+0.146$ when M\&A is coded as success
\end{itemize}

The baseline coding (M\&A = Failure) is defensible because: (1) most M\&A are acqui-hires or asset sales, not billion-dollar exits; (2) M\&A often represents founder exit under duress; (3) the thesis studies growth \emph{capacity} via Later Stage VC, not exit outcomes.

% ============================================================================
\section{Sample Construction Robustness}
\label{sec:robust-sample}
% ============================================================================

\subsection{Survival Bias Assessment}
\label{sec:survival-bias}

The sample requires firms to have breadth scores in both 2021 ($B_0$) and 2025 ($B_T$). This excludes 1,870 firms (1.1\%) lacking $B_T$. Could this exclusion attenuate the core relationships?

\textbf{Method}: Bounds analysis---assign extreme $R$ values to excluded firms:
\begin{itemize}
    \item Lower bound: $R = 0$ (assume all Stayers)
    \item Upper bound: $R = \max(R)$ (assume extreme Movers)
\end{itemize}

\begin{table}[htbp]
    \centering
    \caption{Bounds Analysis for Survival Bias}
    \label{tab:attenuation}
    \begin{tabular}{@{}lrrrl@{}}
        \toprule
        \textbf{Correlation} & \textbf{Restricted} & \textbf{Lower Bound} & \textbf{Upper Bound} & \textbf{Verdict} \\
        \midrule
        $\rho(E,R)$ & $-0.133$ & $-0.131$ & $-0.133$ & Robust \\
        $\rho(R,G)$ & $+0.184$ & $+0.186$ & $+0.174$ & Robust \\
        $\rho(E,G)$ & $-0.042$ & $-0.041$ & $-0.041$ & Robust \\
        \bottomrule
    \end{tabular}
\end{table}

\begin{figure}[htbp]
    \centering
    \includegraphics[width=0.95\textwidth]{img/AppC_Fig_attenuation_test.png}
    \caption{Bounds analysis for survival bias. Restricted sample (black) shows correlations equal to or stronger than full-sample bounds (gray).}
    \label{fig:attenuation}
\end{figure}

\textbf{Result}: The restricted sample shows correlations equal to or stronger than the full-sample bounds. Furthermore, only 3\% of excluded firms are ``Out of Business''---70\% are ``Generating Revenue,'' indicating a data coverage issue rather than survival bias. \textbf{All three hypotheses are robust to survival bias.}

% ============================================================================
\section{Summary of Robustness Tests}
\label{sec:robustness-summary}
% ============================================================================

\begin{table}[htbp]
    \centering
    \caption{Complete Robustness Summary}
    \label{tab:robustness-summary}
    \begin{tabular}{@{}p{3.5cm}cccl@{}}
        \toprule
        \textbf{Test} & \textbf{H1} & \textbf{H2} & \textbf{H3} & \textbf{Notes} \\
        & $\rho(E,R)<0$ & $\rho(R,G)>0$ & $\rho(E,G)<0$ & \\
        \midrule
        \multicolumn{5}{l}{\textit{Measurement}} \\
        \quad Mover threshold ($R>1,5,10$) & \cmark & \cmark & \cmark & MA ranges 1.60--2.60$\times$ \\
        \quad Rank normalization & \cmark & \cmark & \cmark & With threshold definitions \\
        \addlinespace
        \multicolumn{5}{l}{\textit{Outcome definition}} \\
        \quad M\&A = Success ($G=1$) & \cmark & \cmark & \xmark & H3 flips to $+0.146$ \\
        \quad M\&A = Censored & \cmark & \cmark & \cmark & H3 near zero ($-0.002$) \\
        \addlinespace
        \multicolumn{5}{l}{\textit{Sample construction}} \\
        \quad Survival bounds analysis & \cmark & \cmark & \cmark & No attenuation detected \\
        \addlinespace
        \midrule
        \textbf{Overall} & \textbf{Robust} & \textbf{Robust} & \textbf{Sensitive} & H3 depends on M\&A coding \\
        \bottomrule
    \end{tabular}
\end{table}

\textbf{Conclusion}: The Commitment Cage (H1) and Flexibility Flex (H2) are robust across all analytical choices tested. The Funding-Growth Paradox (H3) is robust to measurement and sample construction choices, but sensitive to M\&A outcome coding---when M\&A is coded as success, the paradox reverses. This sensitivity is discussed in Chapter~\ref{ch:conclusion} as a limitation.

\begin{longtable}[]{@{}lcccc@{}}
\caption{Robustness Tests --- Alternative Specifications\textsuperscript{a}}\label{tab:robustness}\tabularnewline
\toprule
Test & \ensuremath{\rho}(E,G)\textsuperscript{b} & \ensuremath{\rho}(E,R) & \ensuremath{\rho}(R,G) & Mover Adv \\
\midrule
\endfirsthead
\toprule
Test & \ensuremath{\rho}(E,G)\textsuperscript{b} & \ensuremath{\rho}(E,R) & \ensuremath{\rho}(R,G) & Mover Adv \\
\midrule
\endhead
Full sample ($N=168{,}011$) & $-0.042$*** & $-0.133$*** & $+0.184$*** & 2.60\ensuremath{\times} \\
2020-2022 cohort & $-0.039$*** & $-0.128$*** & $+0.179$*** & 2.52\ensuremath{\times} \\
2023-2025 cohort & $-0.044$*** & $-0.137$*** & $+0.188$*** & 2.68\ensuremath{\times} \\
Excluding M\&A exits & $-0.002$ & $-0.098$*** & $+0.177$*** & 2.31\ensuremath{\times} \\
Top quartile funding only & $-0.051$*** & $-0.145$*** & $+0.191$*** & 2.78\ensuremath{\times} \\
\bottomrule
\multicolumn{5}{l}{\footnotesize \textsuperscript{a}All correlations Spearman rank; *** $p < 0.001$. See Appendix~C for sensitivity analysis.} \\
\multicolumn{5}{l}{\footnotesize \textsuperscript{b}$\rho(E,G)$ at individual firm level. Industry-level ecological correlations are stronger:} \\
\multicolumn{5}{l}{\footnotesize \phantom{\textsuperscript{b}}Hardware ($-0.11$), Transportation ($-0.10$), but Quantum ($+0.10$). See Table~\ref{tab:industry}.} \\
\end{longtable}



\chapter{Glossary}
\label{app:d}

\section{Core Variables}

\begin{itemize}
    \item \textbf{$C$}: Commitment level (latent; operational vs. vision-level)
    \item \textbf{$E$}: Early-stage funding (\$M USD)
    \item \textbf{$F$}: Strategic Flexibility (latent capability to keep multiple paths viable)
    \item \textbf{$B$}: Strategic Breadth (0--100 vagueness scale); $B_0$ = baseline, $B_T$ = endpoint
    \item \textbf{$R$}: Repositioning $= |B_T - B_0|$ (observable proxy for $F$)
    \item \textbf{$G$}: Growth (binary = 1 if reached Later Stage VC, Series C+)
    \item \textbf{$\mu$}: Belief probability (shared optimism in governance)
    \item \textbf{$\varepsilon$}: Expected belief shift from a signal
\end{itemize}

\section{Key Numbers}

\begin{table}[htbp]
    \centering
    \begin{tabular}{@{}ll@{}}
        \toprule
        Metric & Value \\
        \midrule
        $N$ & 168,011 \\
        $\rho(E,G)$ & $-0.04$ individual; $-0.11$ to $-0.10$ industry-level\textsuperscript{a} \\
        $\rho(E,R)$ & $-0.133^{***}$ (Commitment Cage) \\
        $\rho(R,G)$ & $+0.184^{***}$ (Flexibility Flex) \\
        Mover Advantage & $2.60\times$ ($P(G=1)$: 17.6\% vs 6.7\%) \\
        Stayers / Movers & 61.2\% / 38.8\% \\
        Base growth rate ($G=1$) & 11.0\% \\
        Sweet Spot (Q3) survival & 15.0\% \\
        \bottomrule
        \multicolumn{2}{l}{\footnotesize \textsuperscript{a}Industry-level: Hardware $-0.11$; Transport $-0.10$; Software $-0.00$; Quantum $+0.10$} \\
    \end{tabular}
\end{table}

\textbf{Industry Heterogeneity:}
\begin{table}[htbp]
    \centering
    \begin{tabular}{@{}lrl@{}}
        \toprule
        Industry & $\rho(E,G)$ & Interpretation \\
        \midrule
        Hardware & $-0.108^{***}$ & Cage binds tightest \\
        Transportation & $-0.101^{***}$ & Capital-intensive lock-in \\
        Biotech & $-0.085^{***}$ & High switching costs \\
        Software & $-0.001$ (ns) & Cage releases \\
        Quantum & $+0.095^{*}$ & Era of ferment exception \\
        \bottomrule
    \end{tabular}
\end{table}

\section{Mechanism Terms}

\begin{itemize}
    \item \textbf{Golden Cage}: Structural constraint preventing adaptation due to governance homogeneity; forms through $C \rightarrow E \rightarrow F\downarrow \rightarrow R\downarrow \rightarrow G\downarrow$
    \item \paradox{Funding-Growth Paradox (H3)}: $\rho(E,G) < 0$; early funding correlates negatively with later-stage growth
    \item \cage{Commitment Cage (H1)}: $dR/dE < 0$; funding suppresses repositioning
    \item \flex{Flexibility Flex (H2)}: $dG/dR > 0$; repositioning predicts growth
    \item \textbf{Decomposition}: $dG/dE = (dG/dR) \times (dR/dE) = (+) \times (-) = (-)$; H3 = H2 $\times$ H1
    \item \textbf{Van den Steen Sorting}: Optimists attract optimists; skeptics self-select out, producing belief homogeneity
    \item \textbf{Strategic Ambiguity}: Precision about direction combined with flexibility about destination; attracts diverse believers
    \item \textbf{Belief Homogeneity}: Convergence of beliefs among governance members through sorting
    \item \textbf{Signal Diversity}: Presence of diverse perspectives to interpret market feedback
    \item \textbf{Caged Learning}: Learning ceases when $\mu(1-\mu) < \varepsilon/B$ (Theorem 1)
    \item \textbf{Era of Ferment}: Pre-paradigmatic phase where no dominant design exists; cage releases
    \item \textbf{Mover}: Venture with $R > 0$ (40.3\% of sample)
    \item \textbf{Stayer}: Venture with $R = 0$ (59.7\% of sample)
\end{itemize}

\section{Commitment Types}

\begin{itemize}
    \item \textbf{Vision-level Commitment}: Direction without destination; preserves flexibility (e.g., Tesla: ``accelerating sustainable transport'')
    \item \textbf{Operational Commitment}: Specific technology/market choice; forecloses alternatives (e.g., Better Place: ``battery swapping infrastructure'')
\end{itemize}

\section{Design Principles (Chapter~\ref{ch:design})}

\begin{itemize}
    \item \textbf{Strategic Ambiguity} (\S\ref{sec:strategic-ambiguity}): Commit to direction, not destination. Vision-level commitment attracts diverse believers; operational commitment attracts homogeneous believers.
    \item \textbf{Balanced Growth} (\S\ref{sec:balanced-growth}): Growth = Market $\times$ Ops. Diagnose which bottleneck threatens and fix it before locking in the other dimension. (Diagonal Principle)
    \item \textbf{Financial Vehicle} (\S\ref{sec:financial-vehicle}): Sequence funding sources. Climb the Funding Ladder (government grants $\rightarrow$ matching grants $\rightarrow$ thesis-driven VCs) so flexibility survives until market signals clarify.
    \item \textbf{Symmetry Principle}: Founders should stage operational commitments as VCs stage financial commitments.
    \item \textbf{Preserving Skeptics}: Actively recruit investors with distinct theses; reserve board seats for independent directors; institute red-team decision rules.
\end{itemize}


\chapter{Supplementary Notes}
\label{app:e}

\section{Non-Dilutive Alternatives}

The Quantum exception suggests deep tech ventures may benefit from non-dilutive funding:
\begin{itemize}
    \item \textbf{Government grants}: NSF, DARPA, DOE
    \item \textbf{Strategic partnerships}: Corporate R\&D agreements
    \item \textbf{Prize competitions}: XPRIZE-style awards
\end{itemize}
