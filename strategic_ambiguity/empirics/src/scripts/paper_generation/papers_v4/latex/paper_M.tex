% =============================================================================
% Paper M: Movement Module (Ma + Mb)
% Ma (M12): The Movement Principle - dG/dM > 0
% Mb (M45): The Golden Cage - dM/dF < 0
% =============================================================================

\chapter{Paper M: Movement Matters}
\label{ch:paper_m}

\begin{abstract}
This paper establishes two findings that together explain the Funding Paradox. \textbf{Part Ma} establishes the \textit{Movement Principle}: ventures that strategically reposition ($M > 0$) achieve 18.0\% survival versus 9.9\% for stayers---a $1.82\times$ advantage ($dG/dM > 0$). \textbf{Part Mb} establishes the \textit{Golden Cage}: funding suppresses movement ($dM/dF < 0$), with a one-SD increase in funding predicting 0.4 SD lower movement. The decomposition $dG/dF = (dG/dM)(dM/dF) < 0$ explains why funding hurts growth: funding suppresses the movement that enables growth. Paper V examines \textit{why} funding creates this trap through the heterogeneous effects of initial vagueness ($V_0$).
\end{abstract}

% =============================================================================
% PART Ma: THE MOVEMENT PRINCIPLE (M12)
% =============================================================================

\section{Part Ma: The Movement Principle}
\label{sec:ma}

\begin{center}
\textit{Logic: $dG/dM > 0$ | Anchor Theory: Trajectory $>$ Position | Syntax: El-Zayaty (Empirics)}
\end{center}

\subsection{Abstract}

Does strategic stability predict venture success? Strategy theory suggests yes---commitment creates coordination value. I challenge this assumption by analyzing 408,697 technology ventures from PitchBook (2021--2025). The results establish the \textbf{Movement Principle}: ventures that strategically reposition ($M > 0$) achieve 18.0\% survival versus 9.9\% for stayers---a $1.8\times$ advantage. \textbf{Trajectory matters more than position.}

\subsection{📿 Gospel: The Static Positioning Paradigm}

\citet{porter1996} established the dominant paradigm: ``strategic positions should have a horizon of a decade or more.'' \citet{vandensteen2017} formalized why: ``Strategy is the smallest set of choices to optimally guide other choices.'' \citet{ghemawat1991} concurs: commitment's value stems from its costliness.

\textbf{Null Hypothesis}: Commitment to position creates competitive advantage. If true, stayers should outperform movers.

\subsection{🧩 Puzzle: Why Do Stayers Underperform?}

Yet our data reveal the opposite: Movers achieve 18.0\% survival versus 9.9\% for Stayers---a $1.82\times$ advantage. If commitment to position creates value, why do ventures that reposition dramatically outperform those that stay fixed?

\subsection{🔍 Lens: Trajectory Analysis}

I propose an alternative lens: in high-uncertainty environments, \textbf{trajectory matters more than position}. The object of commitment shifts from static position to movement capacity. This is not anti-strategy---it extends Van den Steen's mechanism by changing \textit{what} merits commitment.

\textbf{Extension}: VdS's commitment value holds, but the \textit{object} of commitment matters. Low uncertainty $\rightarrow$ commit to position. High uncertainty $\rightarrow$ commit to movement.

\subsection{Defining Movement}

Movement $M = |\Delta V|$: magnitude of strategic repositioning measured as absolute change in vagueness between $V_0$ (initial) and $V_T$ (terminal).

Two archetypes emerge:
\begin{itemize}
    \item \textbf{Stayers} ($M \approx 0$): No repositioning
    \item \textbf{Movers} ($M > 0$): Strategic repositioning (either direction)
\end{itemize}

\subsection{Empirical Results}

\begin{table}[htbp]
\centering
\caption{Movement Principle: Stayers vs Movers}
\label{tab:ma_results}
\begin{tabular}{lccc}
\toprule
\textbf{Archetype} & \textbf{Survival} & \textbf{95\% CI} & \textbf{Advantage} \\
\midrule
Stayer & 9.9\% & [9.5\%, 10.3\%] & --- \\
\textbf{Mover} & \textbf{18.0\%} & [17.5\%, 18.5\%] & $\mathbf{1.82\times}$ \\
\bottomrule
\end{tabular}
\end{table}

\textbf{Core Finding}: Movers outperform stayers $1.82\times$. Direction matters little---movement itself dominates. Paper V examines when direction becomes relevant (trap diagnosis).

\subsection{Discussion: Extending Dynamic Strategy}

This finding extends dynamic strategy literature. \citet{eisenhardt1989} showed that fast decision-making predicts performance in high-velocity environments. I show that strategic \textit{movement}---not just decision speed---predicts survival. The static positioning paradigm requires a scope condition: it holds under low uncertainty but inverts under high uncertainty.

% =============================================================================
% PART Mb: THE GOLDEN CAGE (M45)
% =============================================================================

\section{Part Mb: The Golden Cage}
\label{sec:mb}

\begin{center}
\textit{Logic: $dM/dF < 0$ | Anchor Theory: Hannan \& Freeman (Structural Inertia) | Syntax: El-Zayaty (Main Effect)}
\end{center}

\subsection{Abstract}

If movement predicts growth, why don't all ventures move? The answer: \textbf{funding suppresses movement}. A one-SD increase in funding predicts 0.4 SD lower movement ($p < 0.001$). This is the \textbf{Golden Cage}---capital intended to enable experimentation instead constrains it. The mechanism operates through \textbf{structural inertia} \citep{hannan1984}: funding crystallizes organizational structure before strategy can adapt.

\subsection{📿 Gospel: Capital Enables Experimentation}

\citet{camuffo2020} and \citet{nanda2014} established the experimental entrepreneurship paradigm: capital enables experimentation, experimentation enables learning, learning enables growth.

\textbf{Null Hypothesis}: If capital enables experimentation, then $dM/dF \geq 0$. More funding should predict more strategic movement.

\subsection{🧩 Puzzle: Why Does Funding Suppress Movement?}

I find the opposite: $dM/dF < 0$. A one-SD increase in funding predicts 0.4 SD \textit{decrease} in movement. If capital enables experimentation, why do well-funded ventures move \textit{less}?

\subsection{🔍 Lens: Two-Stage Mechanism}

\citet{hannan1984} explain the mechanism. Organizations develop \textbf{structural inertia}---resistance to change that increases with:
\begin{enumerate}
    \item \textbf{Age}: Routines become entrenched
    \item \textbf{Size}: Coordination costs rise
    \item \textbf{Complexity}: Interdependencies constrain change
\end{enumerate}

Funding accelerates all three. Capital enables hiring (size), process formalization (complexity), and milestone commitments (routine entrenchment). The Golden Cage: resources meant to enable adaptation instead crystallize structure.

\textbf{Extension}: Camuffo and Nanda are not wrong---capital enables experimentation \textit{in principle}. But commitment to secure capital constrains experimentation \textit{in practice}. The two-stage mechanism (Commitment$\rightarrow$Capital, then Capital$\rightarrow$Lock-in) explains the apparent contradiction.

\subsection{The Selection-Lock-in Mechanism}

The Golden Cage operates in two stages:

\textbf{Stage 1 (Selection)}: To obtain capital, founders must commit precisely. Investors fund confident visions, not acknowledged uncertainty. Selection filters for precision.

\textbf{Stage 2 (Lock-in)}: Once funded, commitment intensifies through:
\begin{itemize}
    \item[(i)] \textbf{Structural}: Board seats, milestones, reporting requirements
    \item[(ii)] \textbf{Psychological}: Public promises are hard to abandon \citep{staw1976}
    \item[(iii)] \textbf{Social}: Investors who believed the original thesis reinforce it
\end{itemize}

\subsection{Empirical Results}

\begin{table}[htbp]
\centering
\caption{Golden Cage: Main Effect of Funding on Movement}
\label{tab:mb_regression}
\begin{tabular}{lccc}
\toprule
\textbf{DV: Movement (M)} & \textbf{(1)} & \textbf{(2)} \\
\midrule
Funding (F) & $-0.40^{***}$ & $-0.38^{***}$ \\
 & (0.02) & (0.02) \\
\midrule
Controls & No & Yes \\
Industry FE & No & Yes \\
Cohort FE & No & Yes \\
N & 180,994 & 180,994 \\
Adj. $R^2$ & 0.02 & 0.04 \\
\bottomrule
\multicolumn{3}{l}{\footnotesize $^{***}p<0.001$. Robust SE in parentheses.}
\end{tabular}
\end{table}

\textbf{Core Finding}: One SD increase in funding predicts 0.4 SD decrease in movement. The effect is robust to controls and fixed effects. Paper V examines heterogeneity by initial vagueness ($V_0$).

\subsection{Discussion: The Paradox Complete}

The funding paradox is now decomposed:
\begin{equation}
\underbrace{\frac{dG}{dF} < 0}_{\text{Paradox}} = \underbrace{\frac{dG}{dM} > 0}_{\text{Ma: Movement Principle}} \times \underbrace{\frac{dM}{dF} < 0}_{\text{Mb: Golden Cage}}
\end{equation}

Movement helps (Ma). Funding suppresses movement through structural inertia (Mb). But is this effect uniform across all ventures?

\textbf{Handoff to Paper V}: Paper V examines heterogeneity. The funding trap varies by initial vagueness ($V_0$): low-$V_0$ ventures face the commitment trap; high-$V_0$ ventures face the flexibility trap. Both produce stayers through different mechanisms.
