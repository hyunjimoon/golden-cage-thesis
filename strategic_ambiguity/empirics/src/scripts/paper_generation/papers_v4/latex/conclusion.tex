% =============================================================================
% Conclusion (Module C)
% =============================================================================

\chapter{Conclusion: Commit to Movement}
\label{ch:conclusion}

\section{Summary of Findings}

This dissertation investigated a fundamental tension in entrepreneurial strategy: the Funding Paradox. Ventures that raise more funding show lower growth rates ($dG/dF < 0$)---a finding that contradicts the capital-enables-growth assumption underlying both signaling theory and Bayesian entrepreneurship.

\subsection{The Decomposition}

The paradox resolves through decomposition:
\begin{equation}
\underbrace{\frac{dG}{dF} < 0}_{\text{Funding Paradox}} = \underbrace{\frac{dG}{dM} > 0}_{\text{Movement Principle}} \times \underbrace{\frac{dM}{dF} < 0}_{\text{Funding Trap}}
\end{equation}

\subsection{Paper M: Movement}

Paper M established two empirical facts:
\begin{enumerate}
    \item \textbf{MG: Movement Principle} ($dG/dM > 0$): Ventures that strategically reposition achieve 18.0\% survival versus 9.9\% for stayers---a $1.82\times$ advantage.
    \item \textbf{MF: Golden Cage} ($dM/dF < 0$): A one-standard-deviation increase in funding predicts 0.4 SD lower movement. The path $F \rightarrow M \rightarrow G$ explains the Funding Paradox.
\end{enumerate}

\subsection{Paper V: The Learning Trap}

Paper V explained \textit{why} funding suppresses movement:
\begin{itemize}
    \item The \textbf{Learning Trap Condition} $\mu(1-\mu) < \varepsilon/(V+1)$ captures when updating stops
    \item \textbf{High-$V_0$ traps (Flexibility Paradox)}: Unfalsifiable hypotheses prevent learning what to reject
    \item \textbf{Low-$V_0$ traps (Commitment Paradox)}: Belief homogenization eliminates variance needed for updating
    \item \textbf{Heterogeneous effects}: Low-$V_0$ ventures face the commitment trap; high-$V_0$ ventures face the flexibility trap
    \item Funding accelerates both traps through \textbf{stakeholder selection}
\end{itemize}

\subsection{Paper E: The Escape}

Paper E documented \textit{how} ventures escape:
\begin{itemize}
    \item Three escape archetypes: Agile, Forced, Aurora Model
    \item The \textbf{PAE Framework}: Platformize $\rightarrow$ Acculturate $\rightarrow$ Evaluate
    \item Core principle: the more specific your funded vision, the more you need diverse stakeholders who don't fully agree
\end{itemize}

% =============================================================================
% C1: IMPLICATIONS
% =============================================================================
\section{C1: Implications for Practice}
\label{sec:c1_implications}

\subsection{For Entrepreneurs}

\begin{enumerate}
    \item \textbf{Commit to movement, not to position}. The worst outcome is not wrong positioning but static positioning. Among stayers, even the best-positioned quartile achieved only 7.5\% success---substantially underperforming movers in every quartile.

    \item \textbf{Diagnose your trap type}. High-V ventures need focus: force specificity, create testable hypotheses. Low-V ventures need flexibility: preserve belief diversity, add doubters to governance.

    \item \textbf{Manage stakeholder expectations proactively}. The Aurora Model shows that high funding and strategic flexibility can coexist---but only with deliberate expectation management from the start.

    \item \textbf{Build movement infrastructure}. The PAE framework provides operational scaffolding: platformize for structural flexibility, acculturate for cultural permission, evaluate for evidence-based updating.
\end{enumerate}

\subsection{For Investors}

\begin{enumerate}
    \item \textbf{Fund movement capacity, not just vision}. The commitment that predicts success is commitment to adaptation, not commitment to static position.

    \item \textbf{Preserve belief diversity}. When everyone agrees, no one learns. Deliberately include skeptics in governance structures.

    \item \textbf{Design flexible milestones}. Milestone-heavy term sheets show stronger $dM/dF < 0$ effects. Consider milestone structures that reward learning, not just execution.

    \item \textbf{Be willing to be wrong}. Your investment thesis may be wrong. Structure relationships that allow founders to tell you so.
\end{enumerate}

\subsection{For Researchers}

\begin{enumerate}
    \item \textbf{Movement as construct}. This dissertation introduces Movement ($M = |\Delta V|$) as a measurable strategic construct. Future research can extend measurement approaches and examine movement in other contexts.

    \item \textbf{Learning trap mechanism}. The condition $\mu(1-\mu) < \varepsilon/(V+1)$ provides a testable mechanism. Field experiments could directly manipulate belief diversity to test causal effects.

    \item \textbf{Stakeholder selection effects}. The finding that funding selects for stakeholders whose incentives resist movement opens research on investor-founder matching and governance design.
\end{enumerate}

% =============================================================================
% C2: LIMITATIONS AND FUTURE
% =============================================================================
\section{C2: Limitations and Future Directions}
\label{sec:c2_limitations}

\subsection{Correlational Evidence}

The primary limitation is correlational design. The $1.8\times$ movement advantage could reflect:
\begin{itemize}
    \item Reverse causality: successful ventures have resources to reposition
    \item Selection: capable founders both move and succeed
    \item Omitted variables: unobserved factors drive both movement and success
\end{itemize}

We mitigate through cohort fixed effects, propensity score matching, and mechanism analysis. The movement premium persists. But causal claims require experimental validation.

\subsection{Measurement}

Vagueness measurement relies on NLP analysis of company descriptions. Alternative operationalizations may yield different results. We provide robustness checks across measurement approaches.

\subsection{Generalizability}

The sample comprises technology ventures with PitchBook coverage. Findings may not generalize to:
\begin{itemize}
    \item Non-technology sectors with different uncertainty structures
    \item Ventures outside VC ecosystem
    \item Different temporal or geographic contexts
\end{itemize}

\subsection{The New Problem}

Solving the movement problem may create a new problem: coordination failure. If everyone commits to movement, no one commits to anything. The optimal system-level mix of stayers and movers remains unexamined.

\subsection{Future Directions}

\begin{enumerate}
    \item \textbf{Experimental tests}. Field experiments manipulating belief diversity or milestone structure could establish causation.

    \item \textbf{Alternative outcomes}. Beyond Later Stage VC, examine movement effects on profitability, employment, innovation, and acquirer outcomes.

    \item \textbf{Investor behavior}. How do investors respond to movement? Do some investor types reward adaptation while others punish it?

    \item \textbf{Founder heterogeneity}. Which founders move effectively? Serial entrepreneurs? Technically-trained versus business-trained?

    \item \textbf{Optimal timing}. When should ventures move? Early movement may be cheap but uninformed; late movement may be informed but costly.
\end{enumerate}

% =============================================================================
% C3: CODA
% =============================================================================
\section{C3: Commit to Movement}
\label{sec:c3_coda}

We began with a puzzle: why does funding hurt growth? We end with an answer: because funding suppresses the movement that growth requires.

The prescription is clear: \textbf{Commit to movement, not to the promises that fund it.}

Van den Steen was right that commitment creates value. But in high-uncertainty environments, the object of commitment matters as much as the fact of commitment. Commit to learning. Commit to adaptation. Commit to movement.

\begin{center}
\textit{必死卽生, 必生卽死}\\
\textit{(Those who are determined to die will live; those who are determined to live will die.)}\\
\medskip
\textbf{Commit to Movement.}
\end{center}
