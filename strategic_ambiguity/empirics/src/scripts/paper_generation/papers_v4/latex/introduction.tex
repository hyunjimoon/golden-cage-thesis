% =============================================================================
% Introduction - The Funding Paradox (Module I)
% =============================================================================

\chapter{Introduction: The Funding Paradox}
\label{ch:intro}

\section{The Commitment Gospel}

Strategy theory and investment practice share an implicit assumption: commitment attracts capital, capital enables coordination, coordination enables growth. \citet{vandensteen2017} formalizes this logic---strategy creates value precisely because commitment is costly. \citet{ghemawat1991} concurs: irreversibility is not a bug but a feature. The more credibly you commit, the more money you raise, the better you should perform.

This logic has profound implications for entrepreneurial strategy. If commitment creates coordination value, then founders should articulate precise visions, define narrow market categories, and project concrete trajectories. The ``scientific approach'' to entrepreneurship \citep{agrawal2021} extends this prescription: formulate falsifiable hypotheses, gather evidence, and converge on validated positions. Vagueness, in this view, represents noise---an inferior signal that either masks low ability or invites adverse selection.

\section{The Empirical Puzzle}

The data say otherwise. Ventures that raise more funding show \textit{lower} growth rates. The correlation is robustly negative: $\rho(G,F) = -0.196$ ($p < 0.001$, $N = 180,994$). I call this the \textbf{Funding Paradox}.

\begin{equation}
\frac{dG}{dF} < 0
\label{eq:paradox}
\end{equation}

The finding contradicts the capital-enables-growth assumption underlying both signaling theory \citep{spence1973} and the Bayesian entrepreneurship framework \citep{gans2019}. If capital enables experimentation and experimentation enables learning, why would more capital predict less growth?

\section{The Movement Principle}

This dissertation explains why. The answer lies in \textbf{movement}---the magnitude of strategic repositioning a venture undertakes. Movement predicts growth ($dG/dM > 0$), but funding suppresses movement ($dM/dF < 0$).

The decomposition:
\begin{equation}
\underbrace{\frac{dG}{dF} < 0}_{\text{Funding Paradox}} = \underbrace{\frac{dG}{dM} > 0}_{\text{Movement Principle}} \times \underbrace{\frac{dM}{dF} < 0}_{\text{Funding Trap}}
\label{eq:decomposition}
\end{equation}

A positive times a negative yields the paradox.

\section{Core Variables}

I define five variables (see Table~\ref{tab:variables}):

\begin{table}[htbp]
\centering
\caption{Variable Definitions}
\label{tab:variables}
\begin{tabular}{clll}
\toprule
\textbf{Variable} & \textbf{Name} & \textbf{Definition} & \textbf{Operationalization} \\
\midrule
$F$ & Funding & External capital raised & log(\$ raised) \\
$M$ & Movement & Strategic repositioning magnitude & $|V_T - V_0| = |D|$ \\
$G$ & Growth & Venture performance & Later Stage VC (C/D+) \\
$V$ & Vagueness & Strategic ambiguity & Rank-normalized (0--100) \\
$D$ & Direction & Signed repositioning & $V_T - V_0$ \\
\bottomrule
\end{tabular}
\end{table}

Movement is not pivoting randomly. It is disciplined repositioning---what I call \textbf{Parallel Experimentation $\rightarrow$ Strategic Convergence}: ventures initiate multiple concurrent paths, then systematically narrow based on evidence, concentrating commitment on the highest-potential trajectory.

\section{The Null Hypothesis}

The null hypothesis comes from strategy theory. As \citet{porter1996} argues, ``strategic positions should have a horizon of a decade or more, not of a single planning cycle.'' \citet{vandensteen2017} formalizes why: strategy is ``the smallest set of choices to optimally guide other choices''---its value derives from commitment's irreversibility. \citet{ghemawat1991} concurs that ``commitment, far from being a source of inflexibility, is the very essence of strategy.'' All suggest stability should help, not hurt.

My finding qualifies this prescription for entrepreneurial contexts. Here, movers outperform stayers $1.82\times$. The commitment that matters is commitment to \textit{movement capacity}, not to fixed positions.

\section{The Core Finding}

The data reveal a stark pattern:

\begin{table}[htbp]
\centering
\caption{Stayers vs Movers}
\label{tab:archetypes}
\begin{tabular}{lccc}
\toprule
\textbf{Archetype} & \textbf{M} & \textbf{Survival} & \textbf{Advantage} \\
\midrule
Stayer & $\approx 0$ & 9.9\% & --- \\
Mover & $> 0$ & 18.0\% & $1.82\times$ \\
\bottomrule
\end{tabular}
\end{table}

\textbf{Movement matters; direction matters little.} Whether ventures focus (D < 0) or broaden (D > 0), both outperform stayers by similar margins (17.5\% vs 18.4\%). The prescription is clear: move.

\section{The Twin Traps}

Traps occur at both extremes of initial vagueness ($V_0$):
\begin{itemize}
    \item \textbf{High-$V_0$ Trap (Flexibility Paradox)}: Ventures start too vague, unable to focus despite needing precision (mobility sector: $V_0=78$, 91\% stayer ratio, 5\% survival)
    \item \textbf{Low-$V_0$ Trap (Commitment Paradox)}: Ventures start too specific, unable to pivot despite needing flexibility
\end{itemize}

Funding reinforces both traps through stakeholder selection: high-$V_0$ ventures attract ``vision'' investors who reward staying vague; low-$V_0$ ventures attract ``execution'' investors who punish pivoting.

\section{The Prescription}

\textbf{Commit to movement, not to the promises that fund it.}

\section{Dissertation Structure}

This dissertation presents three papers that progressively decompose and explain the Funding Paradox:

\begin{enumerate}
    \item \textbf{Paper M} (Chapter~\ref{ch:paper_m}): Establishes the Movement Principle ($dG/dM > 0$) and the Golden Cage ($dM/dF < 0$). The \textit{what} of the paradox decomposition.
    \item \textbf{Paper V} (Chapter~\ref{ch:paper_v}): Explains the heterogeneous traps by initial vagueness ($V_0$). The \textit{why} through learning trap mechanism.
    \item \textbf{Paper E} (Chapter~\ref{ch:paper_e}): Documents escape strategies through the PAE framework (Platformize $\rightarrow$ Acculturate $\rightarrow$ Evaluate). The \textit{how to escape}.
\end{enumerate}

The Conclusion (Chapter~\ref{ch:conclusion}) synthesizes findings and articulates implications for entrepreneurs, investors, and researchers.
