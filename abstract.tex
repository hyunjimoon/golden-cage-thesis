Venture capital operates on the premise that early capital accelerates growth. Yet analyzing 168,011 startups, I document a \paradox{Funding-Growth Paradox}: early funding correlates \emph{negatively} with later-stage success. At the individual firm level, the correlation is modest ($\rho = -0.04$), reflecting high within-firm variance. At the industry level, the pattern is stark: capital-intensive sectors systematically underperform (Hardware: $\rho = -0.11$; Transportation: $\rho = -0.10$). Why would resources hurt?

The answer lies in governance. To secure funding, founders commit to specific strategies. These commitments attract investors who believe in those strategies; skeptics self-select out. Extending Van den Steen's sorting equilibrium to the investor-founder context, I show that the resulting board lacks cognitive diversity. When market signals suggest pivoting, no one advocates for change. The venture is trapped---not for lack of capital, but for lack of diverse perspectives. I call this the \emph{golden cage}.

Methodologically, I introduce a text-based measure of \textbf{Strategic Breadth ($B$)} to quantify repositioning from company descriptions. The paradox then decomposes into two causal forces. First, the \cage{Commitment Cage}: funding suppresses repositioning ($\rho = -0.133$, $p < 0.001$) because governance homogeneity blocks adaptation. Second, the \flex{Flexibility Flex}: repositioning predicts success, with Movers outperforming Stayers by 2.60$\times$ (17.6\% vs.\ 6.7\%). Together: $dG/dE = (dG/dR) \times (dR/dE) = (+) \times (-) = (-)$. Funding suppresses the very mechanism required for survival.

Industry heterogeneity reveals boundary conditions. The cage binds tightest in capital-intensive sectors where infrastructure investments create switching costs. It releases in pre-paradigmatic sectors like Quantum Computing ($\rho = +0.10$) where no dominant design constrains architectural choices.

To escape the cage, I prescribe three architectural levers: (1) \emph{Scope}---committing to thesis (vision) rather than architecture (solution), attracting diverse believers who agree on direction but differ on mechanisms; (2) \emph{Synchronization}---coordinating capability scaling with market evolution to avoid premature lock-in; and (3) \emph{Sequencing}---ordering capital sources to delay governance homogenization until market signals clarify. This thesis contributes a governance-based theory of venture rigidity and offers design principles for commitment structures that preserve flexibility.
