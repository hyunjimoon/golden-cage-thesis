This thesis studies how venture capital governance, through belief sorting and commitment structures, suppresses the strategic flexibility startups need to survive. The last decade has witnessed hundreds of billions of dollars deployed annually to fuel startup growth. Yet a central puzzle emerges: early funding correlates \emph{negatively} with later success. Analyzing 168,011 U.S.\ ventures (2021--2025), I document this \paradox{Funding-Growth Paradox}: at the firm level, $\rho = -0.04$ ($p < 0.001$); at the industry level, capital-intensive sectors show steeper penalties (Hardware: $\rho = -0.11$; Transportation: $\rho = -0.10$). To address this puzzle, this thesis examines how commitment---the very quality that attracts resources---can foreclose the adaptation that uncertain markets demand. It focuses on two central questions: (1) why does funding suppress flexibility, and (2) how can founders commit credibly while preserving the capacity to pivot?

Part I of this thesis develops the theory and evidence for the \emph{golden cage}---a governance trap where funding homogenizes beliefs. Chapter 2 investigates the mechanism: to secure capital, founders commit to specific strategies; investors who fund share those beliefs; skeptics self-select out. Extending Van den Steen's sorting equilibrium from employee-manager matching to investor-founder contexts, I formalize when organizational learning ceases (\textbf{Theorem 1: Caged Learning}). The resulting board lacks cognitive diversity: when markets shift, no member advocates for change.

Chapter 3 operationalizes the theory through a novel text-based measure of \textbf{Strategic Breadth ($B$)} derived from company descriptions. This metric enables quantification of repositioning---the observable manifestation of flexibility. Chapter 4 tests the mechanism empirically, decomposing the paradox into two causal forces: (i) the \cage{Commitment Cage}, where funding suppresses repositioning ($\rho = -0.133$, $p < 0.001$), and (ii) the \flex{Flexibility Flex}, where repositioning predicts survival---Movers outperform Stayers by 2.60$\times$ (17.6\% vs.\ 6.7\%). Together: $dG/dE = (dG/dR) \times (dR/dE) = (+) \times (-) = (-)$. Industry heterogeneity reveals boundary conditions: the cage binds tightest where switching costs are high (Hardware, Transportation) but releases where no dominant design exists (Quantum: $\rho = +0.10$).

Part II of this thesis turns prescriptive, developing architectural levers for flexibility. Chapter 5 identifies three design dimensions that determine whether ventures can respond to market signals. \emph{Scope} addresses what breadth of commitment to make: commit to thesis (vision) rather than architecture (solution), attracting diverse believers who agree on \emph{why} but differ on \emph{how}. The Tesla-Better Place contrast illustrates the mechanism; the $Q3$ ``sweet spot'' in positioning breadth (15.0\% survival) confirms it empirically. \emph{Synchronization} addresses how to coordinate capability scaling: diagnose binding constraints before locking in the wrong dimension, avoiding the Operational Trap (NxStage) and the Scale Trap (SkinnyGirl). \emph{Sequencing} addresses when to accept different capital sources: climb the funding ladder (grants $\rightarrow$ matching capital $\rightarrow$ thesis-driven VC), delaying governance homogenization until market signals clarify.

This thesis contributes a governance-based theory of venture rigidity and prescribes design principles that preserve flexibility within commitment structures. The findings have implications for founders architecting cap tables, investors structuring syndicates, and policymakers designing innovation ecosystems.
