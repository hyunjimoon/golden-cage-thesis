Venture capital governance, through belief sorting and commitment structures, suppresses the strategic flexibility startups need to survive. The last decade deployed hundreds of billions of dollars annually to fuel startup growth. Yet a central puzzle emerges: early funding correlates negatively with later success. This thesis examines how commitment (the very quality that attracts resources) forecloses the adaptation uncertain markets demand. Two questions guide the investigation: (1) why does funding suppress flexibility, and (2) how can founders commit credibly while preserving pivot capacity?

Chapter 2 develops the \emph{golden cage} theory. To secure capital, founders commit to specific strategies; investors who fund share those beliefs; skeptics self-select out. Extending Van den Steen's sorting equilibrium from employee-manager to investor-founder contexts, I formalize when organizational learning ceases (\textbf{Theorem 1: Caged Learning}). The resulting board lacks cognitive diversity: when markets shift, no member advocates for change.

Chapter 3 operationalizes the theory through Strategic Breadth, a novel text-based measure derived from company descriptions. This metric quantifies repositioning, the observable manifestation of flexibility. Chapter 4 tests the mechanism empirically using 168,011 U.S. ventures, decomposing the paradox into two causal forces: the \cage{Commitment Cage}, where funding suppresses repositioning, and the \flex{Flexibility Flex}, where repositioning predicts survival. Industry heterogeneity reveals boundary conditions: the cage binds tightest where switching costs run high but releases where no dominant design exists.

Chapter 5 turns prescriptive, identifying three design dimensions that determine whether ventures respond to market signals. \emph{Scope} addresses what breadth of commitment to make: commit to thesis (vision) rather than architecture (solution). \emph{Synchronization} addresses how to coordinate capability scaling: diagnose binding constraints before locking in the wrong dimension. \emph{Sequencing} addresses when to accept capital sources: climb the funding ladder from grants through matching capital to thesis-driven VC, delaying governance homogenization until market signals clarify.

This thesis contributes a governance-based theory of venture rigidity and prescribes design principles that preserve flexibility within commitment structures. The findings carry implications for founders architecting cap tables, investors structuring syndicates, and policymakers designing innovation ecosystems.
