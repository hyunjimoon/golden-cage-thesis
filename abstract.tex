Early funding correlates negatively with startup success. This paradox emerges from 168,011 U.S. ventures: the resources meant to fuel growth suppress the flexibility required to survive. The culprit is governance. To secure capital, founders commit to specific strategies; investors who fund share those beliefs; skeptics never join. The resulting board lacks the cognitive diversity to pivot when markets shift. I call this the \emph{golden cage}.

Chapter 2 develops the theory. Extending sorting equilibrium models to investor-founder matching, I formalize when organizational learning ceases (\textbf{Theorem 1}). Chapter 3 introduces Strategic Breadth, a text-based measure that quantifies repositioning from company descriptions. Chapter 4 tests the mechanism: funding suppresses repositioning (the \cage{Commitment Cage}), yet repositioning predicts survival (the \flex{Flexibility Flex}). The cage binds tightest in capital-intensive sectors like Hardware and Transportation, but releases in pre-paradigmatic sectors like Quantum where no dominant design exists.

Chapter 5 turns prescriptive. Tesla committed to "sustainable transport" (thesis-level), attracting diverse believers who supported pivots from Roadster to Model 3. Better Place committed to "battery swapping" (architecture-level), attracting only believers in that mechanism; when charging technology advanced, no board member advocated change. The company liquidated despite raising \$850 million. Three design principles emerge: commit to vision rather than solution (\emph{Scope}), diagnose bottlenecks before scaling (\emph{Synchronization}), and sequence capital sources to delay governance lock-in (\emph{Sequencing}).

This thesis contributes a governance-based theory of venture rigidity and prescribes how founders can commit credibly while preserving the capacity to adapt.
